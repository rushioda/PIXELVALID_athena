% $Id: CoralDB-PixelConn.tex,v 1.5 2008-10-14 20:20:24 jarguin Exp $

\section{Modelling of the Pixel Detector Connectivity}

\subsection{Conventions}

The modelling of the pixel detector connectivity follows roughly the structure of
the Unix file system. The primary structure consists of a folder structure similar
to a set of directories and subdirectories that can be represented by a directed,
acyclic graph. Connections corresponding to this structure are called ``hard links'',
in analogy to the Unix file system.

Since this hierarchical structure is not sufficient, additional arbitrary connections
(``soft links'') are allowed.

In order to distinguish hard and soft links, by convention the terminating slot of a
connection is labelled ``UP'' for hard links, and something other than ``UP'' for soft links.

Table \ref{connextable} shows a simple example of such a structure, and Figure \ref{connexgraph}
shows the corresponding connectivity graph. By convention, solid lines in
connectivity graphs represent hard links, while soft links are shown by dashed lines
(the type names shown in the figure in parentheses beneath the object IDs are not given
in table \ref{connextable}).

\begin{table}[h]
\begin{center}
\begin{tabular}{|l|l|l|l|l|}
\hline
Source ID & Source Slot & Destination ID & Destination Slot & Comment \\ \hline
\hline
PP0S          & D1A\_B02\_S2  & D1A\_B02\_S2     & UP & Hard link\\ \hline
D1A\_B02\_S2\_OB & PP0         & D1A\_B02\_S2     & OB & Soft link \\ \hline
D1A\_B02\_S2    & M1          & D1A\_B02\_S2\_M1  & UP & Hard link\\ \hline
R-OHV3\_1      & SLOT-3      & D1A\_B02\_S2     & HV-B3 & Soft link\\ \hline
\end{tabular}
\caption[Simple connectivity example]
{A few connections for a very simple connectivity example.}
\label{connextable}
\end{center}
\end{table}

\NFIG{h}
{connexgraph.eps}{height=6cm}
{connexgraph}{Connectivity graph for a simple connectivity example}
{Connectivity graph for the simple connectivity example from table \ref{connextable}.}



\subsection{Pixel Detector Connectivity}

%{\it A detailed description of the modelling of the pixel detector connectivity will be added
%to a later version of this note.

%In the mean time, please contact Jean-Francois Arguin (JFArguin@lbl.gov)
%with any questions about the modelling of the pixel detector connectivity.}

%%%%%%%%%%%%%%%%%%%%%%%%%%%%%%%%%%%%%%%%%%%%%%%%%%%%%%%
%
%  PLAN
%
%
%- Mention the concrete usage of the conn DB: DCS, DAQ, book-keeping.
%- Describe pixel detector services in a few sentences
%- Use this paragraph to introduce the different branches (HV, LV...)
%- Give references of connectivity (note they are not necessarily as-built, only the DB is guaranteed to be)
%
%- Mention IDTAG pixel, and examples below are from PIT-ALL-V31
%
%- Describe branches
% * Describe convention for (almost) every object. Describe object in order starting by modules.
% * If trees separate upstream, make new subsection for each separation. 
%
%- Conn DB describe here is essentially frozen
%- Mention the content has been validated and sometimes updated during the service test, SR1 and pit CT
%- Future: add cable number is SLOT to improve the book-keeping purpose of the DB 
%(this way, connection between object closer to actual cables)


The primary clients of CoralDB are the DCS and DAQ software that need connectivity information to operate
and readout the detector. On the DCS side, the content of CoralDB is inspected to create
an xml file \cite{dirk} that is fed into the PVSS-based system used to operate the pixel detector 
services.
%,i.e. the low- and high-voltages of modules and optoboards, interlock signals, cooling loops and environmental sensors. 
On the DAQ side, PixDbInterface (see Sec. \ref{PixDbInterface}) is used to access
connectivity-related information inside the DAQ software, e.g. the correspondance between modules and Readout-Drivers (ROD)
in the VME crates. Furthermore, the pixel connectivity DB is the only complete source of information on how the detector
is connected at a given time. It is thus not only important to make the DCS and DAQ software work, but also as 
an archive of the hardware configuration of the detector services as a function of time.

In this section we describe the content of the pixel connectivity database. We do not aim at 
describing the services of the pixel detector themselves, but rather
their representation in the connectivity DB. General descriptions of the pixel detector services can be found
in \cite{pixelpaper,requirements_services}. More dedicated documents describing specific aspects of the services, 
and which have been used to populate parts of the pixel connectivity DB, can be found in
\cite{atl0092,atl0093,atl0094,atl0102,atl0104,atl0132}\footnote{The detailed connectivity described in these documents 
should not be trusted as up-to date. For instance the pixel connectivity might have been modified after some recent tests have been
performed (e.g. the Service Test and Connectivity Test as mentioned later), and only the connectivity DB is kept 
up-to date at this point. The connectivity DB is thus the only current reliable 
source for the pixel connectivity.}. The rest of the content of the connectivity DB has been 
provided by the specialists of the various areas of the pixel services. 

For clarity of presentation,
we divide the description of the pixel detector connectivity into ``branches''. We will thus present separately 
the pixel services related to the HV, LV, SCO-Link 
(the optoboard power supplies), modules and optoboards NTC signals, DAQ, opto-heaters, cooling loops 
and environmental sensors. For each branch we will describe the list of objects, their naming conventions
and the logic on how they have been connected together. The object type are written in bold character
for easy identification.

The official IDTAG used for the pixel detector connectivity is called ``PIXEL''. At the moment of writing this document,
the most recent connectivity tag was ``PIT-ALL-V32''. It has been used to create the snapshots of the DB shown in 
this section. All the snapshots shows the various services connected to the same example PP0 L0\_B01\_S1\_A6.  
We describe the various objects consisting the connectivity tree starting at the bottom of those DB snapshots.

\subsubsection{HV Branch}
\label{sec:hv_branch}

The HV connectivity branch for L0\_B01\_S1\_A6 is shown in Fig.~\ref{fig:branch_hv}.%
%
\begin{itemize}
\item {\bf Module} (Ex.: ID=L0\_B01\_S1\_A6\_M1A): The ``DCS'' module convention is used as the default module ID. Different
aliases are available under different naming conventions, for instance GEOID (e.g. L0-B01-S01-A/6-M1A), HASHID (e.g. 33742848), 
OFFLINEID (e.g. [2.1.0.0.1.1.0]) and PRODID (e.g. M511634). The modules have no outgoing connections in the connectivity
DB.
\item {\bf PP0} (Ex.: L0\_B01\_S1\_A6): PP0 refers to a ``Patch Panel 0'' row on the Service Quarter Panel (SQP) where 
a half-stave or a disk sector connects to. Again, the DCS naming convention is used by default but two aliases exist.
The GEOID naming convention has the form e.g. SQP-A12-OP-A1-P4L-B and describes the position of the PP0 on the SQP: the letter ``A'' 
in ``A12'' refers to the side and ``12'' to the quadrant, ``OP'' means Outer Panel, ``A1'' describes the octant, ``P4L'' gives
the row number (4) and parity of the panel (L=left), and ``B'' refers to the bottom panel (with respect to ``T'' for top). The other 
naming convention is PRODID that has the form e.g. 14152A.
\item {\bf HV-PP1} (Ex.: PP1B-OSP-A1-R2-P3.4): This object describes the position where the Type 1 and Type 2 cables
connect at PP1. ``OSP'' refers
to the outer service panel, ``A1'' to the octant, ``R2'' to ``Row 2'', ``P3'' to ``Position 3'' and ``.4'' to the label
of the cable connecting at HV-PP4.
\item {\bf HV-PP4} (Ex.: Y2813S2\_HV-PP4-05\_A.O1.4.4): Describes the position of the HV-PP4 channel: ``Y2813S2'' refers
to the rack number, ``05'' to the crate number, ``A'' to the side, ``1'' to the octant, the first 4 in ``4.4''
to the position inside the crate and the second 4 to the cable number.
\end{itemize}

The HV branch separates into two ``sub-branches'' at the node of the object type {\bf HV-PS-CHN}
(see Fig.~\ref{fig:branch_hv}). We describe both sub-branches separately.

\vspace{0.5cm}

\noindent \hspace{0.2cm} {\bf \normalsize PP4 and power supply sub-branch}
%
\begin{itemize}
\item {\bf HV-PS-CHN} (Ex.: Y2813S2\_HV03\_2A.Ch07): The HV power supply channel is a combination of other objects up the tree,
i.e. {\bf HV-PS-RACK} (Y2813S2), {\bf HV-PS-CRATE} (Y2813S2\_HV03), {\bf HV-PS-MOD} (Y2813S2\_HV03\_2A) and finally the 
channel is contained in ``.Ch07''.
\item {\bf HV-CRATE-CONTROLLER} (Ex.: Y2813S2\_C3): It represents the node of the CAN-bus that connects to the {\bf HV-PS-CRATE}.  
\item {\bf CANiseg} (Ex.: HVcrates\_US, Y2813S2\_HV): This is the CAN-bus that controls the HV power supply.  
\item {\bf PVSS-SYSTEM-HV} (Ex.: ATLPIXHV): This object does not represent an actual physical connection but
rather gives the name of the project inside the PVSS-system for the HV system. 
\item {\bf PC-HV} (Ex.: PCATLPIXHV): Describes the computer where the HV PVSS project is installed.
\end{itemize}

\noindent \hspace{0.2cm} {\bf \normalsize Interlock sub-branch}

\vspace{0.3cm}

For more information on the pixel interlock system, see \cite{ilock_matrix}.
%
\begin{itemize}
\item {\bf IDBHV} (Ex.: Y0314S2\_IDB\_HV03): The Interlock Distribution Box (IDB) that routes the interlock signal
created in the Logic Unit (not shown here, see Sec.~\ref{sec:mod_ntc}).
to the corresponding HV power supply channel ({\bf HV-PS-CHN}). 
\item {\bf CAN-FW-ELMB} (Ex.: Y0314S2\_LU): The CAN-bus at the output of the Logic Unit that routes the interlock
signal to the IDB. An interlock signal can come for example from a module temperature exceeding 40$^\circ$C. 
This CAN-bus Y0314S2\_LU appears also in Fig.~\ref{fig:branch_mod_ntc} that describes the services of 
the module NTC signal. We see in Fig~\ref{fig:branch_mod_ntc}, which will be
described in more details later, that the interlock signal is fed into this Logic Unit from the
corresponding BBIM channels of L0\_B01\_S1\_A6. This example illustrates that the interlock
signal links together various branches of the pixel connectivity, more specifically the power branches (HV, LV and SCO-Link)
with the interlock signal branches (e.g. module and optoboard NTC). This CAN-bus is connected to a
PC of type {\bf PC-IDBLU} (ID=PCATLPIXILOCK) that runs the interlock PVSS project {\bf PVSS-SYSTEM-IDBLU}
(ID=ATLPIXILOCK).
\item {\bf CANPSU-IDBLU} (Ex.: EPSU/Y1214A2\_CAN\_PS02/Branch8): This is the branch for the CAN power supply
of type {\bf CAN-PS} (ID=Y1214A2\_CAN\_PS02) that serves the {\bf CAN-FW-ELMB} with ID=Y0314S2\_LU. There are only two CAN-PS for
the whole detector (PS01 and PS02).
\item {\bf CAN-FW-ELMB} (Ex.: BUS\_PSU): The {\bf CAN-PS} itself requires a CAN-bus to be controlled from the outside
world. This is the CAN-bus with ID=BUS\_PSU,
which is itself controlled by the PVSS project ATLPIXPSU of type {\bf PVSS-System-Wiener} installed
on the PC PCATLPIXSP1 of type {\bf PC-VME}. The latter two objects constitutes the top of all the connectivity
branches described in this Section.
\end{itemize}

\begin{figure}
\begin{center}
  %\includegraphics[width=\textwidth]{Branch-HVPP1.epsi}
  \resizebox{17.0cm}{21.0cm}{\includegraphics[width=\textwidth]{Branch-HVPP1.epsi}}
\end{center}
\caption{An illustration of the HV branch}
\label{fig:branch_hv}
\end{figure}

\subsubsection{LV Branch}

The LV branch, which is illustrated in Fig.~\ref{fig:branch_lv}, is largely symmetric to the HV branch. We will only 
describe here the notable differences. 
We note that Fig.~\ref{fig:branch_lv} illustrates only the connectivity for the analog voltage (VDDA), but it is
fully symmetric with the one of the digital voltage (VDD).

The object type {\bf LV-PP2} (ID=PP2\_AP2\_123\_BoardA) constitutes an additional node between PP1 and PP4 with 
respect to the HV branch.
It represents the regulator board located at PP2 that is required for the module low voltage regulation. ``PP2\_AP2\_123'' refers to 
the PP2 crate and ``BoardA'' to the slot in the crate. Two soft-links connect the PP2 board to each module
of L0\_B01\_S1\_A6. The TOSLOT
of these connections contain the type of voltage (VDD or VDDA) and the SLOT the corresponding PP2 channel number.

The object type {\bf LV-PP4} (ID=Y2513S2\_LV\_PP4\_14C.WienerCh2),
which provides a module-level current measurement, is connected on one side to a {\bf LV-PP4-BLOCK}\\ 
(ID=Y2513S2\_LV\_PP4\_14C) that is itself connected to a CAN-bus of type {\bf CAN-PIX-ELMB} (ID=Y2513S2\_LV\_PP4).
The rest of the CAN-bus connectivity is similar to the one described for the HV branch. 

The object {\bf LV-PP4} is connected
on the other side to the power supply {\bf LV-PS}\\ (ID=Y2513S2\_LV14/Output/Channel/9). The latter is
connected to a {\bf LV-PS-CRATE} that is operated via a TCP/IP protocol (object type 
{\bf TCPIP-LV}) since the CAN-bus protocol is not supported by the Wiener power supplies. 

\begin{figure}
\begin{center}
  \resizebox{17.0cm}{21.0cm}{\includegraphics[width=\textwidth]{Branch-LVPP4.epsi}}
\end{center}
\caption{An illustration of the VDDA LV branch. The VDD LV branch is completely symmetric.}
\label{fig:branch_lv}
\end{figure}

\subsubsection{SCO-Link Branch}

Again, the SCO-Link branch is very similar to the LV and HV branches
and we only mention here significant differences. The SCO-Link branch is represented in Fig.~\ref{fig:branch_scolink}.

One notable difference is that the voltages provided by the SCO-Link reach the PP0 via 
the two different PP1 positions {\bf OPTO-PP1} and {\bf LV-PP1}, the latter being the same as for the
LV branch. On one hand, the voltage of the VDC
chip (VVDC) requires regulation and is thus passed from the object type 
{\bf SCO-LINK} through {\bf LV-PP2} before reaching PP1. On the other hand, the voltages
Viset (which controls the optoboard light output power) and V$_{PiN}$ (which drives the DORIC 
chip that transform the light signal from the VME crates into an electrical LVDS signal) do not
need regulation. They pass through two nodes before reaching PP1, {\bf OPTO-PP3-SCO} and
{\bf OPTO-PP2}, where the signal is unaltered but the cable type changes.

\begin{figure}
\begin{center}
  \resizebox{17.0cm}{21.0cm}{\includegraphics[width=\textwidth]{Branch-SCOLink.epsi}}
\end{center}
\caption{An illustration of the SCO-Link branch}
\label{fig:branch_scolink}
\end{figure}

\subsubsection{Module NTC Branch}
\label{sec:mod_ntc}

This branch, shown in Fig.~\ref{fig:branch_mod_ntc}, represents the services needed to readout 
the module NTC and create two types of interlock signals: the standard one (internal to the 
pixel system) and the one for the Bake-Out Box (BOB) that will be sent to the LHC machinerie
during the bake-out of the ATLAS beampipe (for more information on the BOB, see 
\cite{ilock_matrix}). The BOB interlock signal can only be triggered by the B-layer modules.
%
%
\begin{itemize}
\item {\bf OPTO-PP1} (Ex.: PP1B-OSP-A1-R1-P2.24): Describes the PP1 position where the OPTO-NTC Type 2 
cables connect to. The naming convention is described in Sec.~\ref{sec:hv_branch}.
\item {\bf OPTO-PP2-NTC} (Ex.: A.P2-2.31-4-2): This is the passive interface of the NTC signal at PP2. 
The naming convention is the following: ``A.P2-2.31'' gives the PP2 box name, and ``4'' and ``2''
are the input connector group and number, respectively.
\item {\bf BBIM-CHN} (Ex.: Y5904X7\_BBIM05C.Ibox3.Temp7-c): This object represents six channels of the 
Building Block Interlock and Monitoring (BBIM), which provides temperature readout of modules,
optoboards and PP2 regulator stations and participates to building interlock signals. There are 
6 or 7 channels per PP0 depending on the number of modules (``Temp7-c'' refers
to channel 7 to c in hexadecimal numbering).  In the ID name, ``Y5904X7'' gives
the Opto-PP3 rack number, ``05C'' is the crate designator and ``Ibox3'' refers to the PP3 input
connector number.

As can be seen in Fig.~\ref{fig:branch_mod_ntc}, the signal from the BBIM is transfered to three 
different sub-branches that are defined below.

\item {\bf BBIM} (Ex.: Y5904X7\_BBIM05C): The object type {\bf BBIM} simply refers to the BBIM crate. This object
is itself connected to the CAN-bus of type {\bf CAN-PIX-ELMB} with an ID=PP3\_US7. 
The rest of the CAN-bus connectivity is analogous to the one described in Sec.~\ref{sec:hv_branch}.

\item {\bf LUMEor} (Ex.: Y0314S2\_LU07\_Tmod2): This object represents the matrix element of the Logic Unit that
takes the incoming temperature signals from the BBIM and combine them to create the ``standard'' interlock 
signals. The ID is defined as follows: ``Y0314S2'' is the rack number, ``LU07'' is the Logic Unit
number and ``Tmod2'' defines the matrix element of the Logic Unit that corresponds to the PP0 in 
question.

The signals created in the Logic Unit are sent up the tree to the CAN-bus with ID=Y0314S2\_LU and of type
{\bf CAN-FW-ELMB}. We note that this CAN-bus appears also in the HV, LV and SCO-Link branches as can be seen 
in Figs.~\ref{fig:branch_hv}-\ref{fig:branch_scolink}, where the signals from the Logic Unit are sent
to the IDBs and then to the respective power supplies as mentioned previously.

\item {\bf BOB-BLOCK} (Ex.: Y0414S2\_BOB01D): This object represents the block corresponding to this
PP0 in the Bake-Out box. It is connected to the CAN-bus   Y0414S2\_BOB of type {\bf CAN-FW-ELMB}.
Again, the rest of the CAN-bus connectivity is analogous to what has been described previously.
\end{itemize}

\begin{figure}
\begin{center}
  \resizebox{17.0cm}{21.0cm}{\includegraphics[width=\textwidth]{Branch-Opto-PP2-NTC.epsi}}
\end{center}
\caption{An illustration of the module NTC branch}
\label{fig:branch_mod_ntc}
\end{figure}

\subsubsection{Optoboard NTC Branch}

The optoboard NTC branch, shown in Fig.\ref{fig:branch_opto_ntc}, is very similar to the module 
NTC branch and we only mention here the main differences.

The naming convention of the Logic Unit object is different for optoboards than for modules, 
although there is no fundamental reason for that. The object {\bf BBIM-CHN} connects to 
{\bf LUTOpto} with an ID for example like Y0314S2\_LU07\_FPGA/AI/ai\_25.value. The rack and crate 
are the same
as for the module but the matrix element is defined by the part ``ai\_25.value'' in this case.
The next object upstream is of type {\bf LU-FPGA} of which the ID just contains the same rack and 
crate number than for the {\bf LUTOpto} described above (Y0314S2\_LU07\_FPGA). 
The next object upstream is the same CAN-bus than for the module NTC signal.

The main difference for the optoboard NTC branch is that it also feeds an interlock
signal to the Optoboard Heater Interlock (OHI) as can be seen in Fig.~\ref{fig:branch_opto_ntc}
starting from the {\bf LU-FPGA} node.
The signal created in these boxes is used the switch off the optoboard heaters in case of 
optoboards temperature reaching 40$^\circ$.
The corresponding {\bf OHI-BOX-CHN} in our example as the ID Y0314S2\_LU/Y0314S2\_OHIBox02/AI/ai\_6.value
where ``Y0314S2'' is the rack number, OHIBox02 is the box number and ``ai\_6.value'' refers 
to the channel number. The OHI channel is connected through an object of type {\bf OHI-BOX} to 
the same CAN-bus as for the module and optoboard NTC signals.

\begin{figure}
\begin{center}
  \resizebox{17.0cm}{21.0cm}{\includegraphics[width=\textwidth]{Branch-Opto-PP3.epsi}}
\end{center}
\caption{An illustration of the optoboard NTC branch}
\label{fig:branch_opto_ntc}
\end{figure}

\subsubsection{DAQ Branch}

The DAQ connectivity branch is illustrated in Fig.~\ref{fig:branch_daq_optoheater}.

\begin{itemize}
\item {\bf OPTOBOARD} (Ex.: L0\_B01\_S1\_A6\_OB): The optoboards provide the light to electrical 
signal conversion (and vice-versa) for the data-in and out of the modules. There are thus connected to the
ROD and BOC (Back-of-Crate card) located in the VME crates that are described next. The optoboards are also
connected upstream to objects of type {\bf OBLINKMAP}, {\bf CONTAINER} and {\bf ROOT} that do not 
represent physical connections but that are used inside the DAQ software to define the internal 
connectivity based e.g. on the optoboard type (D\_TYPE or B\_TYPE in the SLOT of the {\bf CONTAINER}
to {\bf OBLINKMAP} connection).

The object {\bf OPTOBOARD} is also connected to the Opto-heaters as described in Sec.~\ref{sec:optoheater}.

\item {\bf RODBOC} (Ex.: ROD\_B3\_S8): Represents both the ROD and BOC (that are located in the same
VME slot). In the exmaple ID ROD\_B3\_S8, ``B'' represents the type of module connecting to that {\bf RODBOC} (``B'' for 
B-layer, ``D'' for disk and ``L'' for Layer-1 or -2), ``3'' represents the crate number and ``S8'' 
the slot number in the crate. The connectivity tree separates in two main sub-branches at this node
that we describe separately. There exists aliases of this object under the convention OFFLINEID
and has the form e.g. ``0x130308''.
\end{itemize}

\noindent \hspace{0.2cm} {\bf \normalsize DAQ-related connectivity}

\begin{itemize}
\item {\bf RODCRATE} (Ex.: ROD\_CRATE\_B3): Represents the DAQ crate. The connectivity tree separates
again at this node and connects to the two objects {\bf PARTITION} and {\bf VME-PS}. There
exists two type of aliases for the object {\bf RODCRATE}, one under the convention
HWID that gives the physical location of the crate (e.g. Y.26-05.A2.U07) and the other under
the convention NUMID that gives the DAQ crate number (e.g. ROD\_CRATE\_3).
\item {\bf PARTITION} (Ex.: PIXEL\_B): This object does not represent a physical connection but it is 
used internally in the DAQ software. This object defines the partition in the ATLAS Trigger
and DAQ software to which this PP0 belongs to. The possibilities are PIXEL\_B, PIXEL\_D,
PIXEL\_L and PIXEL\_DET for the B-layer, Disk, Layer-1 and -2, and pixel Local Trigger
Processor (LTP) crates, respectively.
\item {\bf VME-PS} (Ex.: Y2605A2\_VME03): This represents the power supply for the corresponding 
VME crate. It is itself connected by a CAN-bus of type {\bf CAN-FW-Wiener}.
\end{itemize}

\noindent \hspace{0.2cm} {\bf \normalsize BOC monitoring}

\begin{itemize}
\item {\bf BOCMON-DP-ELT} (Ex.: Y2605A2\_BocCr03.Plug13): The temperature of the BOC is monitored
via ELMBs connected to the DCS system. There is a one-to-one correspondance
between the objects of type {\bf BOCMON-DP-ELT} and {\bf RODBOC}
\item {\bf BOC-MONITORING} (Ex.: Y2605A2\_BocCr03): There is one such object in the BOC monitoring
system per crate. The crate number corresponds to the alias NUMID of the object type {\bf RODCRATE} 
described above. It is connected upstream to a CAN-bus of type {\bf CAN-PIX-ELMB} of ID=BocMon. 
The rest of the CAN-bus connectivity has been described previously.
\end{itemize}


\subsubsection{Opto-Heater Branch}
\label{sec:optoheater}

The opto-heater connectivity is shown in the same figure as the DAQ branch
(Fig.~\ref{fig:branch_daq_optoheater}). The first object in the branch, 
{\bf OPTOHEATER}, is connected to {\bf OPTOBOARD} (on the right side of the Fig.~\ref{fig:branch_daq_optoheater}).

\begin{itemize}
\item {\bf OPTOHEATER} (Ex.: A1\_OSP\_BOT): There is one opto-heater for 6 optoboards
at the most (i.e. one per row of an SQP). The ID gives the octant (e.g. A1), the panel on 
the SQP (OSP for outer or ISP for inner) and the location on the outer panel
(TOP or BOT for bottom). There exist one alias for this object under the convention
ATLOFFICIAL that gives the full name of the corresponding data-point element in 
the opto-heater PVSS project (e.g. PIX/OPTOHEATER/SideA/Oct1/OSP\_BOT).
\item {\bf HEATER-PS} (Ex.: ATLPIXLCS7:TEH\_Crate\_06.SW\_Card\_01): This object
represents one of the four opto-heater switching card (Card\_01 to Card\_04). 
\item {\bf HEATER-CC} (Ex.: Y2807S2\_CCard): The four switching cards are then connected
to a controller card, which is itself connected to a CAN-bus of type {\bf CAN-HEATER-ELMB}.
The rest of the CAN-bus connectivity has been described previously.
\end{itemize}

\begin{figure}
\begin{center}
  \resizebox{17.0cm}{21.0cm}{\includegraphics[width=\textwidth]{Branch-Optoboard.epsi}}
\end{center}
\caption{An illustration of the DAQ and Optoheater branch}
\label{fig:branch_daq_optoheater}
\end{figure}

\subsubsection{Environmental Sensors and Cooling Loops Branch}

Humidity and temperature sensors have been installed throughout the pixel package
volume to monitor the environmental conditions of the detector. Some of these sensors are installed
specifically on cooling loops. In this section we described only those sensors connected
to cooling loops but other environmental sensors are readout in exactly the same
fashion. Figure~\ref{fig:branch_env} shows an example, again for L0\_B01\_S1\_A6,
of the connectivity of these objects.

\begin{itemize}
\item {\bf COOLING-LOOP} (Ex.: cooling\_loop14): Each PP0 is connected to one of the
88 pixel cooling loops in CoralDB, which also includes the  optoboard cooling loops. The pixel cooling
connectivity is currently not modelled in more details in CoralDB.
\item {\bf NTC-SENSOR} (Ex.: C1-P5Rs, C1-P5Rex): Each cooling loop is connected to two 
NTC-sensors, one on the supply side (e.g. C1-P5Rs) and one on the exhaust side (e.g. C1-P5Rex).
In the example IDs, ``C1'' refers to the octant where the sensor is located\footnote{In this 
example the sensor is located on the C-side but the PP0 (L0\_B01\_S1\_A6) on the A-side. This can
happen since
each cooling loop serves a bistave and has the supply and exhaust lines located on the same side.}
 and ``P5R'' to the row on the SQP.  Each sensor has an alias under the convention
ATLOFFICIAL (e.g. PIX/ENV/SideC/Oct1/loop014/PP0/P5Rs/Temp) that can yield more information
on the function of the sensor as will be observed below.

As mentioned before, there exists NTC-sensors that are not necessarily connected to cooling loops. 
One example is A4-ISP-BUNDLE-R that has an alias \\ PIX/ENV/SideA/Oct4/INT/ISP\_BUNDLE\_R/Temp.
The alias tells us that the sensor is located between PP0 and PP1 (``INT'' stands for ``intermediate''
region) in the right ISP bundle (ISP\_BUNDLE\_R) on octant C4 (SideA/Oct4).

There exists also humidity sensors of type {\bf H-SENSOR}, for example C2-HYGRO-PP1-TOP with 
an alias PIX/ENV/SideC/Oct2/PP1/HYGRO/Top. The alias name tells us for instance that it is
of type Hygrotron (HYGRO, other possibilities are Honeywell (HWELL) and Xeritron (XER)) 
and is located in the top region at PP1.

\item {\bf ENV-PP1-PIN} (Ex.: PP1B-ISP-C1-R2-P1.1/LEMO-F\_PIN-31): Each sensor is connected to
three objects of type {\bf ENV-PP1-PIN}, each representing a pin in the environmental cable
connecting at PP1. The SLOT of these connection tells the type of pin (SUPPLY, RETURN or DRAIN). The first
part of the ID (PP1B-ISP-C1-R2-P1.1) yields the PP1 position (as defined above) and the last
part gives the pin number on the LEMO-F connector.

Each sensor is connected directly to the channel of the Building Block Monitoring (BBM) used
to read it out, which will be described later.

\item {\bf ENV-PP1} (Ex.: PP1B-ISP-C1-R2-P1.1): The PP1 position of the environmental Type 2 cable.
\item {\bf ENV-PP2} (Ex.: PP2\_CP2\_241-1.1): The PP2 crate number and position where the 
environmental cable connects to.
\item {\bf BBM-NTC-CHN} (Ex.: Y5904X7\_BBM02B.ADCPlug0X.Tx4): The BBM channel used to readout
this NTC sensor. The naming convention is the same as for the BBIM described in 
Sec.\ref{sec:mod_ntc}. The analogous object type for humidity sensors is {\bf BBM-HS-CHN}.
\item {\bf BBM-NTC-BOX} (Ex.: Y5904X7\_BBM02B.ADCPlug0X.Tx0-6): Describes the BBM box that serves channel 0-6.
\item {\bf BBM} (Ex.: Y5904X7\_BBM02B): The BBM crate number located at PP3.
\item {\bf CAN-PIX-ELMB} (Ex.: PP3\_US7): The CAN-bus used to readout and control the BBM. The rest of the 
CAN-bus connectivity is completely analogous to the one described in Sec.~\ref{sec:hv_branch}.
\end{itemize}

\begin{figure}
\begin{center}
  \resizebox{17.0cm}{21.0cm}{\includegraphics[width=\textwidth]{Branch-ENV.epsi}}
\end{center}
\caption{An illustration of the cooling loop and environmental sensor branch}
\label{fig:branch_env}
\end{figure}

The content of pixel connectivity DB is mostly complete at the time
of writing this document. The assignment between the service channels and readout units
(i.e. modules, optoboards, environmental sensors) has been established in several tests like the 
Service Test \cite{service_test} and the SR1 \cite{sr1_ct} and Pit Connectivity Tests.
Some additions are still planned, for instance to add the Read-Out System (ROS) in the DAQ 
branch. We also plan to introduce the cable numbers in the connectivity DB
in the SLOT or TOSLOT (which currently is often a dummy assignment). The cable number is not
required to make the DAQ or DCS software work but will be useful to archive completely 
the pixel connectivity inside CoralDB.

