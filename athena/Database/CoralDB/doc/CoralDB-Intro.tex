% $Id: CoralDB-Intro.tex,v 1.7 2008-03-03 21:39:27 beringer Exp $

\section{Introduction}

% Main author: Juerg

The ATLAS pixel detector consists of 1744 pixel modules with a total
of 80 million readout channels that are arranged in 3 barrel layers
plus 3 disk layers on each side. In order to operate and readout the
pixel detector, its modules must be connected to low- and high-voltage
power supplies, to cooling, to the DAQ system, the DCS system and the
interlock system using several thousand electrical and optical
cables, resulting in O($10^4$) connections one needs to keep track of.
The necessary information about these connections is stored in the
connectivity database.

The connectivity information is logically part of the pixel
configuration data. While the connectivity data allows only to
determine for example to which high-voltage channel a given pixel
detector module is connected, the configuration data also specifies
the actual voltage that should be applied to this module. The complete
configuration data (including connectivity) should provide all the
information necessary to power up, configure and operate the pixel
detector in a given configuration. However, the configuration data
does not specify at what times (or for which runs) a given detector
configuration was used. This is the role of the conditions database
that in general will contain a reference to a given version (or tag)
of the configuration data, possibly together with a copy of some
relevant data items such as applied voltages.

Over the course of the experiment, many different detector
configurations will be used. Once used for taking data to be analyzed
later, the configuration information must be preserved. Unless one
wants to copy all of the configuration data into the conditions
database, the configuration database (and thus the connectivity
database) must support some versioning with the possibility to lock a
given version to prevent it from later modification.

The pixel connectivity database is implemented based on CoralDB, a
generic connectivity database application that was developed
specifically for the pixel detector but could equally be used for managing
connectivity information anywhere else in Atlas.  As the name
indicates, CoralDB is based on CORAL \cite{CORAL} and can thus
use different relational databases to store the connectivity
information, including in particular Oracle databases and SQLite
files.

The main features of CoralDB are:
\begin{itemize}
\item Generic connectivity database (ie independent of pixel detector)
\item Modelling of connectivity based on a list of connections between objects
\item Each object can have an arbitrary number of payload data fields attached
\item Objects may be given a type to distinguish different classes of objects
\item In addition to a primary object name, each object may have one or more
  aliases for different naming conventions
\item Each connectivity database may store multiple versions (tags) of connectivity.
\item Different versions of connectivity, aliases or payload data may or may not share the same set of
  objects. This means one can store the connectivity of completely
  unrelated systems in the same Oracle database without any naming conflicts or other interference.
\item Versioning of the actual connectivity, the payload data, and aliases is
  independent (i.e.\ one doesn't need to generate a new version of the connectivity
  information if some of the payload data or an alias is changed)
\item Versions can be locked to prevent accidental modification
\item Simple mechanism to keep track of what version was considered "current" at
  any given time (i.e.\ IOV for tags)
\item Web interface to search and browse a connectivity database using any of
  the available naming schemes, and to create connectivity graphs
\item Web interface to generate spreadsheet-like tables from the database
\item Command-line utilities to create databases, manage tags, search the
  database, enter data, etc.
\item C++ API
\item Read-only Perl API
\end{itemize}

In this note the design and implementation of CoralDB, as well as the
modelling of the specific pixel detector connectivity are
described. Specific recipes on how to perform certain operations with
CoralDB on the pixel connectivity database are given in section
\ref{HOWTO}.

