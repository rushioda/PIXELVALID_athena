\documentclass[11pt]{article}
\newdimen\SaveHeight \SaveHeight=\textheight
\textwidth=6.5in
\textheight=8.9in
\hoffset=-.5in
\voffset=-1in
\def\topfraction{1.}
\def\textfraction{0.}           

\begin{document}
\title{TruthHelper -- Predicates for selecting particles\\
release 5.3.0 and later\\ 
January 29, 2003}
\author{ Ian Hinchliffe (I\_Hinchliffe@lbl.gov),\\
         updated by Ewelina Lobodzinska (ewelina@mail.desy.de)}

\maketitle   

The installed library   TruthHelper in
Generators/GenAnalysisTools/TruthHelper
 provides a set of predicates that can be used in order to select
 partilces of certain types. It runs against the HepMC events that are
 produced either by GeneratorModules or GenzModule and hides the
 complex status codes from the user.

The available predicates are as follows


\begin{itemize}
\item IsGenerator  returns true on particles that were part of the
  event that came from the event generator. Partilces produced by
  Geant are returned false
\item IsGenNonInteracting returns true on neutrinos and other non
  interacting particles such as those from supersymmetry. It returns a
  complete list including any geant secondaries. Is has more partilces
  than results from the combination of IsGenStable and a subseqent
  selection on the IsGenNonInteracting PDG ID codes.
\item IsGenStable returns  true on all stable particles. In the case of events
  from GenzModule these are all the partilces that are passed as input
  to Geant as either non-interacting or as particles that are
  subsequently interact and decay in Geant
\item IsGenSimulStable returns true on all stable particles during generation and simulation
\item IsGenInteractinging is a subset of IsGenStable that excludes non
  interacting particles. It includes electrons, photons, muons, and
  hadrons that are considered as imput to geant such as pions, kaons
  and $K_L$.
\item IsGenType allows a list of PDG ID codes to be specified and
  returns true on all the partilces in the record (independent of
  their status codes) that have these the specifed code (or its
  negative) For example code 11 would return all electrons  and
  positrons including
  those from Geant showers. Code 24 would find all the W bosons,
  (including any maked as documentaries which may not be physical)
\item IsConversion returns true for the particles that convert into the $e^+e^-$ pair
\item isBremsstrahlung returns true for the input particles in the Bremsstrahlung process
\end{itemize}

An example of the use of these predicates can be found in
GeneratorModules/src/TruthDemo.cxx which makes a histogram of all the
stables and shows that the total energy of these is the CM energy of
the PP collision. The critical code parts are
\begin{verbatim}
StatusCode TruthDemo::initialize(){
......
  m_tesIO = new GenAccessIO();
\end{verbatim}
which provides access to the event

 \begin{verbatim}
StatusCode TruthDemo::execute() {
...
 float totenergy = 0.;
  float pxbalance = 0.;
  float pybalance = 0.;
    // Iterate over MC particles  We are using the IsGenStable predicate from
    IsGenStable ifs;
  vector<const HepMC::GenParticle*> particles;
  StatusCode stat = m_tesIO->getMC(particles, &ifs);
  for (vector<const HepMC::GenParticle*>::iterator pitr = particles.begin();
       pitr != particles.end(); pitr++) {          
    pxbalance += (*pitr)->momentum().x();
    pybalance += (*pitr)->momentum().y();
    totenergy += (*pitr)->momentum().e();
\end{verbatim}
which is summing up the momentum and energy of all the stables. 
 
\end{document}




