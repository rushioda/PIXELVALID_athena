\documentclass[11pt]{article}
\newdimen\SaveHeight \SaveHeight=\textheight
\textwidth=6.5in
\textheight=8.9in
\textwidth=6.5in
\textheight=9.0in
\hoffset=-.5in
\voffset=-1in
\def\topfraction{1.}
\def\textfraction{0.}   
\def\topfraction{1.}
\def\textfraction{0.}           
\begin{document}
\title{McAtNlo\_i: An interface between MC@NLO and Athena}
\author{  Ian Hinchliffe (I\_Hinchliffe@lbl.gov), Borut Kerevan (borut.kersevan@cern.ch) and Georgios Stavropoulos (George.Stavropoulos@cern.ch) }
%\today

\maketitle           

This package runs MC@NLO  from within Athena. \\See the example
in {\bf McAtNlo\_i/share/jobOptions.McAtNloHerwig.py } which show how to
read MC@NLO events and hadronize them using Herwig.

Users must first run 
MC@NLO in standalone mode and make a file of events. An athena job
then takes these events hadronizes them and passes them down the
Athena event chain. The events must be made with a version of MC@NLO
that is compatible, recent versions that support the Les Houches
interface should be acceptable. The current implementation is compatible
with the version 3.1 format.

To hadronize {\bf MC@NLO} generated events with Herwig, you only need to run athena with the jobOptions
file jobOptions.McAtNloHerwig.py by typing in the prompt \\
{\it athena.py jobOptions.McAtNloHerwig.py}\\

More infomation about MC@NLO here

http://www.hep.phy.cam.ac.uk/theory/webber/MCatNLO/\\

{\large \bf Note on LHApdf stucture functions (release 11.0.0)}

In the case you want to run MC@NLO with the LHAPDF structure functions you need to
set the autpdf variable to HWLHAPDF and the modpdf one to the LHAPDF set/member index
(see the documentation of the Generators/Lhapdf\_i package for the LHAPDF set/member
index settings). Up to v 3.1, MC@NLO is using the PDFLIB and not the LHAPDF one. In this
case you need to edit the inparmMcAtNlo.dat and the event files and replace the PDFLIB
structure functions set/member with the LHAPDF one. You can have a look at the
McAtNlo\_i/share/inparmMcAtNlo.dat and McAtNlo\_i/share/tt.events files, as an example
of what needs to be changed.\\

{\large \bf Note on the new interface for version 3.3.1}
 
There present interface works for MC@NLO up to version 3.3.1 (at least) but \emph{should} be 
backwards compatible down to version 3.1. All failues \emph{must} be reported (if you want to get
it fixed, of course).

\end{document}
