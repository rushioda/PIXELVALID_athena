\documentclass[a4paper,12pt]{article}

\usepackage{latexsym}
\usepackage{graphics}
\usepackage{epsfig}
\usepackage{multirow}
\usepackage{amssymb}
\usepackage{times}


% etc
\setlength\topmargin{10pt}
\setlength\oddsidemargin {-30pt}
\setlength\evensidemargin{-30pt}
\setlength\textwidth{500pt}
\renewcommand{\floatpagefraction}{0.95}
\renewcommand{\topfraction}{0.9}
\renewcommand{\textfraction}{0.1}

\DeclareOldFontCommand{\tt}{\normalfont\sf\small}{\mathtt}
\renewcommand{\tablename}{{\bf Table}}
\renewcommand{\thetable}{{\bf \arabic{table}}}
\renewcommand{\figurename}{{\bf Fig.}}
\renewcommand{\thefigure}{{\bf \arabic{figure}}}
\renewcommand{\thefootnote}{\fnsymbol{footnote}}
\renewcommand{\arraystretch}{0.9}
\newcommand{\authorrunning}[1]{\edef\authorrun{#1}}
\newcommand{\titlerunning}[1]{\edef\titlerun{#1}}
\usepackage{fancyhdr}
\cfoot{}\rfoot{}\lfoot{}
\lhead{\sffamily\small \authorrun:}
\chead{\hspace{4.5cm}\sffamily\small \titlerun}
\rhead{\sffamily\small \thepage}
\pagestyle{fancy}

\newcommand{\athena}{{\tt ATHENA }}
\newcommand{\acermc}{{\tt AcerMC }}
\newcommand{\pythia}{{\tt PYTHIA }}
\newcommand{\herwig}{{\tt HERWIG }}
\newcommand{\tauola}{{\tt TAUOLA }}
\newcommand{\photos}{{\tt PHOTOS }}
\newcommand{\sherpa}{{\tt SHERPA }}

\begin{document}


\titlerunning{TAUOLA and PHOTOS\ldots}

\authorrunning{{B. P. Kersevan,E. Richter-Was}}

\title{Processing generated events with TAUOLA and PHOTOS using the Athena interface}

\author{
Borut Paul Kersevan \\
\small Faculty of Mathematics and Physics, University of Ljubljana,
 Jadranska 19, SI-1000 Ljubljana, Slovenia. \\ \small and  \\
\small  Jozef Stefan Institute, Jamova 39, SI-1000 Ljubljana, Slovenia.\\[10pt]
El\. zbieta Richter-W\c{a}s\\
\small CERN, EP Division, 1211 Geneva 23, Switzerland, \\ \small and\\
\small Institute of Computer Science, Jagellonian University, 30-072 Krakow,
 ul. Nawojki 11, Poland.\\ \small and\\
\small Institute of Nuclear Physics, 31-342 Krakow, ul. Radzikowskiego 152, Poland.
}

\maketitle

\abstract{\it This note describes the basic steps of how to process the events hadronised by either 
\pythia, \herwig, \sherpa or {\tt ParticleGenerator} with \tauola and/or \photos by using the ATLAS \athena interface.}

%\begin{picture}(0,0)
%\put(300,480){\large ATL-COM-SOFT-2003-007}
%\end{picture}
\newpage
\boldmath
%%%%%%%%%%%%%%%%%%%%%%%%%%%%%%%%%%%%%%%%%%%%%%%%%%%%%%%%%%%%%%
\section{Where to Start}
%%%%%%%%%%%%%%%%%%%%%%%%%%%%%%%%%%%%%%%%%%%%%%%%%%%%%%%%%%%%%%
\unboldmath

This note describes the basic steps of how to produce simulated events
in which the $\tau$-lepton decays are handled by \tauola \cite{TAUOLA}
and the final state QED radiation of leptons and hadrons is added by
the \photos generator \cite{PHOTOS}. Both \tauola and
\photos interfaces to the \athena setup are described.

In what follows the detailed guideline for the two interfaces is provided.

\boldmath
%%%%%%%%%%%%%%%%%%%%%%%%%%%%%%%%%%%%%%%%%%%%%%%%%%%%%%%%%%%%%%
\section{Using \tauola and \photos in the \athena framework}
%%%%%%%%%%%%%%%%%%%%%%%%%%%%%%%%%%%%%%%%%%%%%%%%%%%%%%%%%%%%%%
\unboldmath
\enlargethispage{0.8cm}

\boldmath 
%%%%%%%%%%%%%%%%%%%%%%%%%%%%%%%%%%%%%%%%%%%%%%%%%%%%%%%%%%%%%%%%%%%
\subsection{Installing and Running}
%%%%%%%%%%%%%%%%%%%%%%%%%%%%%%%%%%%%%%%%%%%%%%%%%%%%%%%%%%%%%%%%%%%%
\unboldmath

There are certainly many ways of how the user can set up his athena framework
using the {\tt cmt} tool; the user is referred to:
\begin{center}
{\it http://atlas.web.cern.ch/Atlas/GROUPS/SOFTWARE/OO/Development/SwDevUserGuide/userguide.html}
\end{center}
and 
\begin{center}
{\it http://atlas.web.cern.ch/Atlas/GROUPS/SOFTWARE/OO/tools/cmt/Tutorial.html}
\end{center}
for generic up-to-date {\tt cmt \& athena} instructions and to
\begin{center}
{\it http://www-theory.lbl.gov/~ianh/monte/Generators/}
\end{center}
for the specific {\tt Generators} package information.

The easiest way to set up the {\tt athena} environment is to {\it checkout} the
{\tt TestRelease} package:
\begin{center}
{\tt cmt co TestRelease}
\end{center}
in the user's cmt development directory\footnote{The user might want subsequently to
modify the {\tt TestRelease} {\tt requirements} file in order not to link the
entire {\tt ATLAS} software setup but only the required packages. It is not
mandatory but the {\tt athena} running time gets somewhat shorter.}. After
performing the usual:
\begin{itemize}
\item {\tt cd Testrelease/TestRelease-xx-xx-xx/cmt}
\item {\tt source setup.sh}
\item {\tt gmake}
\item {\tt cd ../run}
\end{itemize}
the user should have the setup ready to run the \athena interface.
\newpage

A number of {\tt jobOptions} files which enable one to process the events in the \athena
framework using \pythia, \herwig or \sherpa  with \tauola and/or \photos is provided, the names
should be self-explanatory:
\begin{center}
{\tt jobOptions.Pythia.AtlasTauola.txt},
{\tt jobOptions.Pythia.AtlasTauolaPhotos.txt},\\
{\tt jobOptions.Herwig.AtlasTauola.txt},
{\tt jobOptions.Herwig.AtlasTauolaPhotos.txt},\\
{\tt jobOptions.ParticleGenerator.AtlasTauola.txt}\ldots

\end{center}

The files should serve as templates that can be expanded and modified
further according to the purpose/scope of the user. The switches used
for steering \tauola and/or
\photos are described in section  Section \ref{s:steering}.

\boldmath 
%%%%%%%%%%%%%%%%%%%%%%%%%%%%%%%%%%%%%%%%%%%%%%%%%%%%%%%%%%%%%%%%%%%
\subsection{Details of the \athena interface}
%%%%%%%%%%%%%%%%%%%%%%%%%%%%%%%%%%%%%%%%%%%%%%%%%%%%%%%%%%%%%%%%%%%%
\unboldmath

The \tauola and \photos are interfaced to \athena through the modules {\tt
Tauola\_i} and {\tt Photos\_i} residing in the {\tt Generators} package. 
The passing of the {\tt HEPEVT} event record
between various modules (e.g. \pythia and {\tt Tauola\_i}, \herwig and {\tt
Photos\_i} etc..) is done by a new {\tt Atlas\_HEPEVT} class which in stored
to and/or read from the StoreGate persistency during the consecutive steps in
accordance with the ATLAS programming strategies. The sequence is sketched in
the following diagram:

\begin{center}
\epsfig{file=acdia_athena_atlas2.eps,width=0.65\linewidth}
\end{center}

In addition, the interface to ParticleGenerator (generating single taus) is now in place.
The HepMC entry is in this case used to fill the HEPEVT event record using the HEPEVT::Wrapper
class methods and the AtlasHEPEVT class is subsequently used to pass it around as in \pythia case. 

Furthermore \sherpa is now interfaced to \tauola and \photos; this is done by a dedicated 
converter from HepMC to HEPEVT data format using the the HEPEVT::Wrapper
class methods. Note that the various \sherpa processes still need some checking; it has been verified to work with the jets+Z-boson processes.

\clearpage
\boldmath
%%%%%%%%%%%%%%%%%%%%%%%%%%%%%%%%%%%%%%%%%%%%%%%%%%%%%%%%%%%%%%
\subsection{Steering \tauola and \photos \label{s:steering}} 
%%%%%%%%%%%%%%%%%%%%%%%%%%%%%%%%%%%%%%%%%%%%%%%%%%%%%%%%%%%%%%
\unboldmath

It is highly recommended that in case \tauola is used for $\tau$ decays that
\photos is also activated and the setting {\tt photos pmode} is set at least to {\tt 2}
in order to keep the $\tau$-leptons radiating (see below), since it gives 
the necessary contribution to the expected $\tau$ branching ratios.

\vspace{0.2cm}\noindent
The switches and settings specific to the \tauola library are set in the 
{\tt Tauola.TauolaCommand} vector in the {\tt jobOptions} file :
\begin{itemize}
\item{\tt  tauola polar} (D=1)\ : \ Polarisation switch for tau decays \\
{\tt tauola polar 0} - switch polarisation off \\
{\tt tauola polar 1} - switch polarisation on 
\item{\tt  tauola radcor} (D=1)\ : \ Order(alpha) radiative corrections for tau decays\\
{\tt tauola radcor 0} - switch corrections off\\
{\tt tauola radcor 1} - switch corrections on
\item{\tt  tauola phox} (D=0.01)\ : \ Radiative cutoff used in tau decays\\ 
{\tt tauola phox 0.01} - default value by TAUOLA authors
\item{\tt  tauola dmode} (D=0)\ : \ Tau and tau pair decay mode\\
{\tt tauola dmode 0} - all decay modes allowed\\
{\tt tauola dmode 1} - (LEPTON-LEPTON): only leptonic  decay modes\\
{\tt tauola dmode 2} - (HADRON-HADRON): only hadronic  decay modes\\
{\tt tauola dmode 3} - (LEPTON-HADRON): one tau decays into leptons and the other one into hadrons\\
{\tt tauola dmode 4} - ($\tau \to \pi \nu$)  : taus are restricted to decay to a pion and neutrino
\item{\tt  tauola jak1/tauola jak2}\ : \ Decay modes of taus according to charge, the list is
taken from the \tauola output. The listing gives only $\tau^-$ modes, the
$\tau^+$ are charge conjugate, neutrinos are omitted. {\tt tauola jak1}\ : \  decay
mode of $\tau^+$, {\tt tauola jak2}\ : \  decay mode of $\tau^-$:\\
{\tt tauola jak1/2  1}   - $\tau^-  \to  e- $\\
{\tt tauola jak1/2  2}   - $\tau^-  \to  \mu-$  \\
{\tt tauola jak1/2  3}   - $\tau^-  \to  \pi^-$ \\
{\tt tauola jak1/2  4}   - $\tau^-  \to  \pi^-, \pi^0 $\\
{\tt tauola jak1/2  5}   - $\tau^-  \to  A_1^-$ (two subch)\\
{\tt tauola jak1/2  6}   - $\tau^-  \to  K^- $\\
{\tt tauola jak1/2  7}   - $\tau^-  \to  K^{*-}$ (two subch)\\
{\tt tauola jak1/2  8}   - $\tau^-  \to  2\pi^-,  \pi^0,  \pi^+$\\
{\tt tauola jak1/2  9}   - $\tau^-  \to  3\pi^0,        \pi^-$\\
{\tt tauola jak1/2  10}  - $\tau^-  \to  2\pi^-,  \pi^+, 2\pi^0$ \\
{\tt tauola jak1/2  11}  - $\tau^-  \to  3\pi^-, 2\pi^+$\\
{\tt tauola jak1/2  12}  - $\tau^-  \to  3\pi^-, 2\pi^+,  \pi^0 $\\
{\tt tauola jak1/2  13}  - $\tau^-  \to  2\pi^-,  \pi^+, 3\pi^0$\\
{\tt tauola jak1/2  14}  - $\tau^-  \to  K^-, \pi^-,  K^+$\\
{\tt tauola jak1/2  15}  - $\tau^-  \to  K^0, \pi^-, \bar{K^0} $\\
{\tt tauola jak1/2  16}  - $\tau^-  \to  K^-,  K^0, \pi^0 $\\
{\tt tauola jak1/2  17}  - $\tau^-  \to  \pi^0  \pi^0   K^-$\\
{\tt tauola jak1/2  18}  - $\tau^-  \to  K^-  \pi^-  \pi^+ $\\
{\tt tauola jak1/2  19}  - $\tau^-  \to  \pi^- \bar{K^0}  \pi^0  $\\
{\tt tauola jak1/2  20}  - $\tau^-  \to  \eta  \pi^-  \pi^0$\\
{\tt tauola jak1/2  21}  - $\tau^-  \to  \pi^-  \pi^0  \gamma$\\
{\tt tauola jak1/2  22}  - $\tau^-  \to  K^-  K^0$

\item{\tt missing boson identity}\ : \ In case of \sherpa and some other generators the
original boson entry might be missing from the event record; in order
to use proper polarisation information for tau pair production the
boson is ambiguous and might be supplied by hand; by default ID is set
to Z$^0$=23:\\ {\tt tauola boson ID} (D=23).
\end{itemize}

\vspace{0.2cm}\noindent
The switches and settings specific to the \photos routines are set are set in the 
{\tt Photos.PhotosCommand} vector in the {\tt jobOptions} file:
\begin{itemize}
\item{\tt  photos pmode} (D=1)\ : \ Radiation mode of photos:\\
{\tt photos pmode 1} - enable radiation of photons for leptons and hadrons\\
{\tt photos pmode 2} - enable radiation of photons for $\tau$-leptons only\\
{\tt photos pmode 3} - enable radiation of photons for leptons only
\item{\tt  photos xphcut} (D=0.01)\ : \ Infrared cutoff for photon radiation:\\
{\tt photos xphcut 0.01} - default value by PHOTOS authors
\item{\tt  photos alpha}\ : \ Alpha(QED) value used in \photos:\\
{\tt photos alpha $\rm <$ 0} - leave default (D=0.00729735039)
\item{\tt  photos interf} (D=1)\ : \ Photon interference weight switch\\ 
{\tt photos interf 1} - interference is switched on\\
{\tt photos interf 0} - interference is switched off
\item{\tt  photos isec} (D=1)\ : \ Double bremsstrahlung switch:\\
{\tt photos isec 1} - double bremsstrahlung is switched on\\
{\tt photos isec 0} - double bremsstrahlung is switched off
\item{\tt  photos itre} (D=1)\ : \ Higher bremsstrahlung switch:\\
{\tt photos itre 1} - higher bremsstrahlung is switched on\\
{\tt photos itre 0} - higher bremsstrahlung is switched off
\item{\tt  photos iexp} (D=1)\ : \ Exponential bremsstrahlung switch:\\
{\tt photos iexp 1} - exponential bremsstrahlung is switched on\\
{\tt photos iexp 0} - exponential bremsstrahlung is switched off
\item{\tt  photos iftop} (D=0)\ : \ Switch for $gg(qq) \to t\bar{t}$ process radiation:\\ 
{\tt photos iftop 1} - the procedure is is switched on\\
{\tt photos iftop 0} - the procedure is is switched off
\end{itemize}

Specific information produced by \tauola and \photos is stored in respective
files {\sf atlastauola.out} and {\sf atlasphotos.out}. A point to stress is that
in case the tau decay was restricted the hard process cross-section given by the
main generator (\pythia, \herwig, \acermc \ldots) should be multiplied by a
branching ratio as detailed at the end of {\sf atlastauola.out}, for example:
\begin{verbatim}
         ------< TAUOLA BRANCHING RATIO FOR TAU DECAYS >------
           THE TAU DECAYS ARE RESTRICTED TO A:
 
                    LEPTON-HADRON  DECAY MODE

          THE PROCESS CROSS-SECTION MUST BE MULTIPLIED BY:

           -> A BRANCHING RATIO =   0.459303E+00 FOR TWO TAUS
 
          IN THE HARD PROCESS DECAY PRODUCTS!!!
          ------> TAUOLA BRANCHING RATIO FOR TAU DECAYS <------
\end{verbatim}

In addition further settings are needed for \pythia and \herwig steering:
\begin{itemize}
\item  When using \tauola for $\tau$ decays in \pythia the $\tau$ decays have to be
switched off by adding:\\[3pt]
\hspace*{2cm} {\tt "pydat3 mdcy 15 1 0",}\\[3pt]
to the {\tt Pythia.PythiaCommand} vector. 
\item  When using \photos for generation of radiation off leptons in \pythia the
\pythia lepton radiation has to be switched off by adding:\\[3pt]
\hspace*{2cm}{\tt "pydat1 parj 90  20000.",}\\[3pt]
to the {\tt Pythia.PythiaCommand} vector. 
\item   When using \tauola for $\tau$ decays in \herwig the \herwig process has to be
notified by adding:\\[3pt]
\hspace*{2cm}{\tt "taudec TAUOLA",}\\[3pt]
to the {\tt Herwig.HerwigCommand} vector.
\item Since no lepton QED radiation for photons is implemented in \herwig no
additional switches are necessary.
\end{itemize}

\boldmath 
%%%%%%%%%%%%%%%%%%%%%%%%%%%%%%%%%%%%%%%%%%%%%%%%%%%%%%%%%%%%%%%%%%%
\section{Technical description of \tauola and \photos interfaces to \pythia and \herwig} 
%%%%%%%%%%%%%%%%%%%%%%%%%%%%%%%%%%%%%%%%%%%%%%%%%%%%%%%%%%%%%%%%%%%%
\unboldmath


\boldmath 
%%%%%%%%%%%%%%%%%%%%%%%%%%%%%%%%%%%%%%%%%%%%%%%%%%%%%%%%%%%%%%%%%%%
\subsection{Interface of \tauola to \pythia, \herwig} 
%%%%%%%%%%%%%%%%%%%%%%%%%%%%%%%%%%%%%%%%%%%%%%%%%%%%%%%%%%%%%%%%%%%%
\unboldmath

\begin{itemize}
\item{\tt Interface to \pythia}:
\end{itemize}

The choice of \tauola library in \pythia sets the $\tau$-s to be stable by
setting {\tt MDCJ(15,1)=0}, thus leaving them to be treated by \tauola
procedures. The
\tauola library is called via the {\tt TAUOLA\_HEPEVT(IMODE)} routine, where {\tt
IMODE=-1,0,1} represent the initialisation, operation and finalisation
respectively. In the operation mode the call to {\tt TAUOLA\_HEPEVT(0)} is made
after the {\tt PYEVNT} and {\tt PYHEPC(1)} calls which generate the full event and
translate and fill it into the {\tt HEPEVT} common block. Tauola now decays the previously
undecayed resonances so no subsequent calls to \pythia are needed. It was
decided against using the provided \pythia interface via the {\tt PYTAUD}
routine since this routine is called for each occurrence of a $\tau$-lepton
separately so no (complex) polarisation options can be included without
substantial changes to \pythia routines.
%Pythia is finally used to decay the unstable resonances left undecayed by Tauola. 

\begin{itemize}
\item{\tt Interface to \herwig}:
\end{itemize}
The present version 6.5 of \herwig contrary to \pythia does internal tracking of
the polarisations of the $\tau$-leptons for the more complex built-in processes
(e.g. Higgs decays); contrariwise nothing can be done for the external hard
processes passed to \herwig for hadronisation. In order to preserve this feature
the {\tt HWDTAU} routine has been modified to perform the \tauola calls only for
internal \herwig processes which provide the $\tau$ polarisation information; other
$\tau$-lepton decays are again executed via a call to {\tt TAUOLA\_HEPEVT}. Since
\herwig is using the {\tt HEPEVT} common block already as the internal event
record no translation of events is needed. In order to also keep other \herwig
parameters current (e.g. {\tt IDHW} of the particle) a call to a new routine 
{\tt HWHEPC} is provided; this new routine also corrects the vertex positions of
the $\tau$ decay products and sets the status codes of new particles to the
\herwig recognisable values. Tauola now decays the previously
undecayed resonances so no subsequent calls to \herwig are needed. 
%Subsequently a call to {\tt HWDHAD} is made in
%order to decay any undecayed particles.

\boldmath 
%%%%%%%%%%%%%%%%%%%%%%%%%%%%%%%%%%%%%%%%%%%%%%%%%%%%%%%%%%%%%%%%%%%
\subsection{Interface of \photos to \pythia and \herwig} 
%%%%%%%%%%%%%%%%%%%%%%%%%%%%%%%%%%%%%%%%%%%%%%%%%%%%%%%%%%%%%%%%%%%%
\unboldmath

\begin{itemize}
\item{\tt Interface to \pythia}:
\end{itemize}

Choosing \photos to provide the QED final state radiation sets the parameter
{\tt PARJ(90)= $2 \cdot 10^{4}$} order to prevent \pythia to radiate photons off leptons,
thus inducing double counting. The parameter is representing the threshold in
GeV below which leptons do not radiate. Pythia does not contain the routines to
handle radiation of hadrons however some caution is necessary due to some
exceptions/specific decays (e.g. $\pi^0 \to e^+ e^- \gamma$) which are 
generally recognised by \photos itself. The initialisation, execution and finalisation
are done by calls to the new {\tt PHOTOS\_HEPEVT(IMODE), IMODE=-1,0,1} subroutine, constructed
for this purpose by the authors. Since \photos operates on the {\tt HEPEVT}
record the calls to {\tt PYHEPC} routine are again necessary.

\begin{itemize}
\item{\tt Interface to \herwig}:
\end{itemize}
The present version 6.5 of \herwig does itself not provide the final state QED
radiation of any kind, thus the inclusion of \photos is simple and possibly also
rather necessary. The call sequence is the same as in \pythia, with no
need for event record conversion. 


\boldmath 
%%%%%%%%%%%%%%%%%%%%%%%%%%%%%%%%%%%%%%%%%%%%%%%%%%%%%%%%%%%%%%%%%%%
\subsection{Details of the \tauola implementation}
%%%%%%%%%%%%%%%%%%%%%%%%%%%%%%%%%%%%%%%%%%%%%%%%%%%%%%%%%%%%%%%%%%%%
\unboldmath

The \tauola library is built from the latest distribution source with the {\it
Cleo} setup option. The native random generator was replaced with a link to the
\athena random generator thus the corresponding random seeds/streams need to be
initialised in the \athena interface.

The {\tt TAUOLA\_HEPEVT} routine is a modification
of the original {\tt TAUOLA} routine provided in the file {\sf tauface\_jetset.f}
in the \tauola distribution. The modifications were restricted to allow for the
'overloaded' use of the HEPEVT record by \pythia and \herwig, where respective mother
and daughter pointers (which should match) sometimes point to different
particles (e.g. hard process copies and similar). The original {\tt TAUOLA}
routine already worked when interfaced to \pythia, albeit requiring the
non-default setting of {\tt MSTP(128)=1}, whereby all the (sometimes useful)
links to the hard record were lost. The modified version also works with the
default \pythia {\tt MSTP(128)=0} setting. 

Additional modifications were made in
order to use the parameters set in {\sf tauola.card} and the call to {\tt
INIMAS} routine was replaced by a call to {\tt TAU\_INIMAS}, which is a copy of
the former but sets the particle masses to the values of \pythia or \herwig
defaults. 

A routine {\tt TAUBRS} that defines the special decay modes (e.g. LEPTON-HADRON,
where one tau decays hadronically and the other one leptonically) was written;
it operates by modifying the internal {\tt GAMPRT} array on an event by event
basis. A related routine {\tt TAUBR\_PRINT} prints the value of the branching
ratio into the {\sf tauola.out} file.No changes to the native tauola code was 
necessary (and therefore not made). 

\boldmath 
%%%%%%%%%%%%%%%%%%%%%%%%%%%%%%%%%%%%%%%%%%%%%%%%%%%%%%%%%%%%%%%%%%%
\subsection{Details of the \photos implementation}
%%%%%%%%%%%%%%%%%%%%%%%%%%%%%%%%%%%%%%%%%%%%%%%%%%%%%%%%%%%%%%%%%%%%
\unboldmath

The \photos is also built from the latest distribution source. As in \tauola The
native random generator was replaced with a link to the
\athena random generator thus the corresponding random seeds/streams need to be
initialised in the \athena interface.

The {\tt PHOTOS\_HEPEVT} routine and the subsequently called {\tt
PHOTOS\_HEPEVT\_MAKE} are newly written routines that inspects the {\tt HEPEVT}
record for charged particles and finds their highest 'mother' particle, which is
consequently passed to \photos routines as the starting point. \photos itself then
walks down the branches and performs radiation where possible. A bookkeeping of
the starting points is made in {\tt PHOTOS\_HEPEVT} in order to prevent multiple
invocations of radiation on the same particle.

Original \photos code had to be modified due to the 'overloaded' {\tt HEPEVT}
record, since its requirements for matching mother-daughter pointers were too
strict for either \pythia (with external processes and/or {\tt MSTP(128)=0}
setting) or \herwig. The modification was limited to {\tt PHOTOS\_MAKE} and {\tt
PHOBOS} routines. In addition the tracking of {\tt IDHW} array was added to {\tt
PHOTOS\_MAKE} to accommodate the \herwig event record. A further modification was
however necessary in the {\tt PHOIN} routine since in \herwig the entry {\tt
JMOHEP(2,I)} is not empty but filled with colour flow information, which in turn
inhibited \photos radiation off participating particles\footnote{\photos expects
the non-zero second 'mother' {\tt JMOHEP(2,I)} entry only for $gg(qq) \to
t\bar{t}$ process, which is treated by a set of dedicated routines; this in turn
clashes with \herwig 'overloaded' {\tt JMOHEP(2,I)} entry.}. 

%%%%%%%%%%%%%%%%%%%%%%%%%%%%%%%%%%%%%%%%%%%%%%%%%%%%
\begin{thebibliography}{99}


\bibitem{TAUOLA}
S. Jadach, J. H. Kuhn, Z. Was, Comput. Phys. Commun. {\bf 64} (1990) 275;
M. Jezabek, Z. Was, S. Jadach, J. H. Kuhn, Comput. Phys. Commun. {\bf 70}
(1992) 69; R. Decker, S. Jadach, J. H. Kuhn, Z. Was, Comput. Phys. Commun. {\bf 76} 
(1993) 361.  

\bibitem{PHOTOS}
E. Barberio and Z. Was, Comp. Phys. Commun. {\bf 79} (1994) 291.

\bibitem{athena_ac}
B. Kersevan, E. Richter-Was, ATLAS Internal Note, ATL-COM-SOFT-2003-001 (2003).

\end{thebibliography}

\end{document}